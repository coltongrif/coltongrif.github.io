\documentclass{article}
\usepackage{akwon}
\setlength{\parindent}{0pt}
\title{Homework 1 feedback}
\author{14/20}
\date{}
\begin{document}
\maketitle
\begin{enumerate}
	\item (1/2) I don't recognize the graph that was drawn on the page, was it drawn horizontally? Even then, it still doesn't look quite right\ldots I suspect a typo was made when the function was input to a graphing calculator.\\
		The function $f(x) = e^{x^{2}} \sin(4x)$ \emph{is} odd, because $f(-x) = e^{(-x)^{2}} \sin(4(-x)) = e^{x^{2}} (- \sin(4x)) = - f(x)$.
	\item (1/2) For (b), the function $\cos(x^{5})$ is \emph{not} invertible, in particular because the graph does not pass the horizontal line test on the domain $[0, \pi/2]$. Another way to see this is that the largest possible range for the ``inverse'' $\sqrt[5]{\cos^{-1}(y)}$ is $[0, \sqrt[5]{\pi/2}]$, but $\sqrt[5]{\pi/2} \approx 1.09$. That means that there are values of $x$ between $1.09$ and $\pi/2 \approx 1.57$ such that $g(f(x)) \neq x$, since $g$ can only ever output a value that is at most $\sqrt[5]{\pi/2}$.\\
		For (c), the function is \emph{not} invertible because the domain is too large; again, the easiest way to see this is that the graph of $\sin(x)$ fails the horizontal line test on the domain $[0, \pi]$. However, if we were to restrict the domain to $[0, \pi/2]$, then the function would be invertible.\\
		For (d), I don't understand the comment about the function increasing on $(1, \infty)$. I think the function is always decreasing on $[0, \infty)$.
	\item Good!
	\item (1/2) The correct answer was given for (a) using a valid technique, but the purpose of the homework is to illustrate how to use the techniques from that week of instruction. L'Hospital's rule has not been introduced yet.\\
		Insufficient detail provided for (d). 
	\item (1/2) For (a), it appears the limit was written incorrectly; the denominator is missing?\\
		For (c), what does it mean to cancel out $(x-1)$ from the numerator and denominator? (The $x-1$ in the numerator was the input to the sine function.) What does $\frac{\sin^{2}}{x-1}$ mean?
	\item Good!
	\item For (a), to say that there is a discontinuity at $x=1$ but then the function can be rewritten so that it is continuous is actually the definition of a removable discontinuity. In other words, ``there was a discontinuity (when the function had $\frac{x^{3} - 1}{x-1}$), but now (when the function is rewritten as $x^{2} + x + 1$) there isn't; it was \emph{removed}.'' This is why we call these points removable discontinuities.
	\item (1/2) $f(-2) = 32$, so there was a typo somewhere. However, just calculating $f(2) = -16$ is not enough, because this only demonstrates that $f$ changed sign once; this only guarantees the function has at least one root. The question asks for more than one root. One way to do this is to check $f(-1.5) < 0$, so that there is a root between $-2$ and $-1.5$, and then check $f(1) = 2 > 0$, so there is a root between $1$ and $2$. Thus, there are at least \emph{two} roots. (Also, the given polynomial $f$ has degree 8, not degree 3, so the statement about odd polynoimals always having roots does not apply.)
	\item (1/2) For (a), the way the argument should be presented for full credit is something like: define $f(x) = 2023 - \sin(x) - x$. Then, $f(0) = 2023 > 0$, but $f(2025) < 0$. Therefore, by the intermediate value theorem, $f$ has a root between $0$ and $2023$. This gives a solution to $2023 - \sin(x) = x$. (I do not understand what ``$2023 - \sin(x)$ covers all real numbers'' means, because this expression can only take values in $2023 \pm 1$, since $\sin(x)$ is between $-1$ and $1$.)\\
		For (b), no additional points deducted because this was my mistake, but there is actually a typo ($\operatorname{exp}(2023x) = x$ actually does not have a solution, but $\operatorname{exp}(2023x) = -x$ does; this is what I meant to write).\\
		For (c), the reasoning is insufficient. A valid argument would take the form of the one in (a), using the intermediate value theorem as was done in class. For example, defining $f(x) = \sin(x) - x^{2023}$ and checking $f(1) > 0$, $f(-1) < 0$ guarantees there is a root of $f$ between $-1$ and $1$. A root of $f$ is the same as a solution to $\sin(x) = x^{2023}$.
\end{enumerate}
\end{document}
