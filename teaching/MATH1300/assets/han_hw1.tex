\documentclass{article}
\usepackage{akwon}
\setlength{\parindent}{0pt}
\title{Homework 1 feedback}
\author{}
\date{16/20}
\begin{document}
\maketitle
\begin{enumerate}
	\item (1/2) $f(-x)$ should be $e^{(-x)^{2}} \sin(4(-x)) = e^{x^{2}} (- \sin(4x))$. It seems that inputting $-x$ into the sine term was neglected.
	\item (1/2) For (c), $\sin(x)$ does not pass the horizontal line test on the domain $[0, \pi]$, but only the domain $[0, \pi/2]$.\\
		For (d), $\frac{1}{1+x^{2}}$ \emph{does} pass the horizontal line test on the domain $[0, \infty)$ although it fails the horizontal line test on $(-\infty, \infty)$.
	\item Good!
	\item Good!
	\item (1.5/2) For (d), I don't understand why $\lim_{x \to 1} \sin(x) \approx 0.0174524$. When I compute it, I get $0.84147\cdots$. The limit should just evaluate ``as-is'' to $\frac{\sin(\sin(1))}{\sin(1)}$, because nothing ``goes wrong''.
	\item Good!
	\item (1.5/2) For (c), there is definitely a discontinuity at $x=3$, because there is no term in the numerator $x^{4} + x^{2} + 5x$ to ``cancel out'' with $x-3$. Thus, as $x$ gets closer and closer to 3, $\frac{x^{4} + x^{2} + 5x}{x-3}$ will approach $+\infty$ (from the right) or $-\infty$ (from the left).
	\item (1/2) The question asked for proof that there is \emph{more} than one root; to get full credit, we need evidence that the function changes sign twice or more. This requires checking points inside the interval, not just the endpoints. For example, one has $f(-2) > 0$, $f(-1.5) < 0$, and so by the IVT there is a root between $-2, -1.5$. Then, $f(1) > 0$ and $f(2) < 0$, and so there is another root between $1$ and $2$. Thus there is more than one root.
	\item For (b): there is no penalty for this, because the problem actually had a typo in it (and so with the function I gave, there actually was no root); however, it is not true that $f(-1) < 0$ and $f(1) > 0$, and so the IVT does not apply. 
\end{enumerate}
\end{document}
