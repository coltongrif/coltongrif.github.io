\documentclass{article}
\usepackage{akwon}
\setlength{\parindent}{0pt}
\title{Homework 3 feedback}
\author{20/20}
\date{}
\begin{document}
\maketitle
\begin{enumerate}
	\item Good!
	\item Good!
	\item In the same way that we can find the derivative of $\operatorname{arctan}(x)$ is $\frac{1}{1+x^{2}}$, we can find $\operatorname{arccot}(x)$ is $\frac{-1}{1+x^{2}}$. If we let $y = \operatorname{arccot}(x)$ then since $1 + \cot^{2} y = \csc^{2}y$, we have that $\csc^{2} y = \csc^{2}(\operatorname{arccot} x) = 1 + x^{2}$.
	\item Good!
	\item Good!
	\item Good!
	\item Good! This is not necessary for this problem, but to see why a function with $f''(x) > 0$ and $f(x) < 0$ cannot exist, notice that a function that is concave up (everywhere) must always stay strictly above any of its tangent lines. However, basically any tangent line to the graph must eventually go above the $x$-axis, hence $f$ must also eventually be above the $x$-axis.
	\item Good!
	\item Good!
\end{enumerate}
\end{document}
