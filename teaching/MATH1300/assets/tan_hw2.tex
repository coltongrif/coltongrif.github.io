\documentclass{article}
\usepackage{akwon}
\title{Homework 2 feedback}
\author{15/20}
\date{}
\begin{document}
\maketitle
\begin{enumerate}
	\item (0/2) For (b), the product rule must be used. This means the derivative is computed as
		\[
			(x)' \sin x + x (\sin x)' = \sin x + x \cos x,
		\]
		and then substituting $x = \pi/3$ gives the answer $\sin(\pi/3) + \pi/3 \cos(\pi/3) = \frac{\sqrt{3}}{2} + \frac{\pi}{3} \cdot \frac{1}{2}$.\\
		For (c), the power rule does not apply for exponential functions, only monomials (such as $x^{2}$, which is totally different from $2^{x}$). The way we computed the derivative of $2^{x}$ in class was by writing it in terms of $e^{x}$ and using the chain rule:
		\[
			2^{x} = (e^{\log 2})^{x} = e^{x \log 2},
		\]
		and then the derivative of this is $e^{x \log 2} \cdot \log 2 = 2^{x} \log 2$. So, the answer should be $16 \log 2$.
	\item Good!
	\item Good!
	\item There are some arithmetic errors in (b): the denominator is $(1+x)^{2}$, not $(1 + x^{2})$ which was written in the quotient rule calculation. Also, the derivative of $\cos(x)^{3}$ is $3 \cos(x)^{2} \cdot (- \sin(x))$, not $- \sin(3x)$. 
	\item Good!
	\item (1/2) There are several arithmetic errors/miscopies from line to line. The implicit differentiation is done correctly, but most of the calculations after that don't make sense to me. When substituting $x = 1, y = -1$, we should get
		\[
			\cos(-\pi) \pi (-1 + y') = \pi(1 + y')
		\]
		and then
		\[
			\pi - \pi y' = \pi + \pi y',
		\]
		so that $y' = 0$. 
	\item There is a typo in the derivative of $\operatorname{arccos}$, since it should be $\frac{-1}{\sqrt{1 - x^{2}}}$, but it appears that $\frac{-1}{\sqrt{1 + x^{2}}}$ was used instead? Otherwise, all looks good.
	\item (0/2) I don't understand the steps taken. It is true that $2x$ is the derivative of $x^{2}$, but this does not immediately lead to $y = 2x-1$ being the correct line. Here is a full solution.\\

		The equation for the tangent line at a given point $(a,a^{2})$ is
		\[
			(y - a^{2}) = 2a(x - a) \Leftrightarrow y = 2ax - a^{2}.
		\]
		Thus, in order for the line to pass through $(0, -1)$, we must have $-a^{2} =  -1$, i.e. $a^{2} = 1$ and so $a = \pm 1$. Then we find that the two tangent lines that pass through $(0, -1)$ are $y = 2x - 1$, $y = -2x - 1$.
\end{enumerate}
\end{document}
