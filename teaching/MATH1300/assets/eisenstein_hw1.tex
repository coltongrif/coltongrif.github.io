\documentclass{article}
\usepackage{akwon}
\setlength{\parindent}{0pt}
\title{Homework 1 feedback}
\author{}
\date{19/20}
\begin{document}
\maketitle
\begin{enumerate}
	\item Good!
	\item (1/2) For (b), the fact that $g(x) = \sqrt[5]{\cos^{-1}(x)}$ has largest possible domain $[-1, 1]$ isn't the best explanation for why $f$ isn't invertible. If we restricted the domain to $[0,1]$ then there isn't really an issue; the bigger problem is that the image of $g$ on the domain $[0,1]$ is $[0, \sqrt[5]{\pi/2}] \approx [0, 1.09]$, which is much smaller than $[0, \pi/2]$. This means that for $x$ between $1.10$ and $\pi/2$, $g(f(x)) \neq x$, since $g(f(x))$ is always in $[0, \sqrt[5]{\pi/2}]$. (This is basically the same as observing that the graph of $f(x)$ does not pass the horizontal line test on the interval $[0, \pi/2]$.)\\
		Similar comments for (c).
	\item Good!
	\item Good!
	\item Good!
	\item Good!
	\item Good!
	\item Good!
	\item There should be a better argument for (c) also using the intermediate value theorem\ldots
\end{enumerate}
\end{document}
